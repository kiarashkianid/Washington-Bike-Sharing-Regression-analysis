% Options for packages loaded elsewhere
\PassOptionsToPackage{unicode}{hyperref}
\PassOptionsToPackage{hyphens}{url}
%
\documentclass[
  10pt,
]{article}
\usepackage{amsmath,amssymb}
\usepackage{iftex}
\ifPDFTeX
  \usepackage[T1]{fontenc}
  \usepackage[utf8]{inputenc}
  \usepackage{textcomp} % provide euro and other symbols
\else % if luatex or xetex
  \usepackage{unicode-math} % this also loads fontspec
  \defaultfontfeatures{Scale=MatchLowercase}
  \defaultfontfeatures[\rmfamily]{Ligatures=TeX,Scale=1}
\fi
\usepackage{lmodern}
\ifPDFTeX\else
  % xetex/luatex font selection
\fi
% Use upquote if available, for straight quotes in verbatim environments
\IfFileExists{upquote.sty}{\usepackage{upquote}}{}
\IfFileExists{microtype.sty}{% use microtype if available
  \usepackage[]{microtype}
  \UseMicrotypeSet[protrusion]{basicmath} % disable protrusion for tt fonts
}{}
\makeatletter
\@ifundefined{KOMAClassName}{% if non-KOMA class
  \IfFileExists{parskip.sty}{%
    \usepackage{parskip}
  }{% else
    \setlength{\parindent}{0pt}
    \setlength{\parskip}{6pt plus 2pt minus 1pt}}
}{% if KOMA class
  \KOMAoptions{parskip=half}}
\makeatother
\usepackage{xcolor}
\usepackage[margin=1in]{geometry}
\usepackage{longtable,booktabs,array}
\usepackage{calc} % for calculating minipage widths
% Correct order of tables after \paragraph or \subparagraph
\usepackage{etoolbox}
\makeatletter
\patchcmd\longtable{\par}{\if@noskipsec\mbox{}\fi\par}{}{}
\makeatother
% Allow footnotes in longtable head/foot
\IfFileExists{footnotehyper.sty}{\usepackage{footnotehyper}}{\usepackage{footnote}}
\makesavenoteenv{longtable}
\usepackage{graphicx}
\makeatletter
\def\maxwidth{\ifdim\Gin@nat@width>\linewidth\linewidth\else\Gin@nat@width\fi}
\def\maxheight{\ifdim\Gin@nat@height>\textheight\textheight\else\Gin@nat@height\fi}
\makeatother
% Scale images if necessary, so that they will not overflow the page
% margins by default, and it is still possible to overwrite the defaults
% using explicit options in \includegraphics[width, height, ...]{}
\setkeys{Gin}{width=\maxwidth,height=\maxheight,keepaspectratio}
% Set default figure placement to htbp
\makeatletter
\def\fps@figure{htbp}
\makeatother
\setlength{\emergencystretch}{3em} % prevent overfull lines
\providecommand{\tightlist}{%
  \setlength{\itemsep}{0pt}\setlength{\parskip}{0pt}}
\setcounter{secnumdepth}{-\maxdimen} % remove section numbering
\usepackage{helvet}
\ifLuaTeX
  \usepackage{selnolig}  % disable illegal ligatures
\fi
\usepackage{bookmark}
\IfFileExists{xurl.sty}{\usepackage{xurl}}{} % add URL line breaks if available
\urlstyle{same}
\hypersetup{
  pdftitle={Should Washington DC's Bikeshare System Reduce Operations During the Winter Months?},
  pdfauthor={Kiarash Kianidehkordi, Min Ji Koo, Joyce Lin},
  hidelinks,
  pdfcreator={LaTeX via pandoc}}

\title{Should Washington DC's Bikeshare System Reduce Operations During
the Winter Months?}
\author{Kiarash Kianidehkordi, Min Ji Koo, Joyce Lin}
\date{2024-12-04}

\begin{document}
\maketitle

\section{Contributions}\label{contributions}

\begin{itemize}
\tightlist
\item
  \textbf{Kiarash Kianidehkordi}: introduction, analysis code,
  interpretation
\item
  \textbf{Min Ji Koo}: introduction, analysis code
\item
  \textbf{Joyce Lin}: introduction, methods, analysis code, knitting and
  finalizing
\end{itemize}

\begin{center}\rule{0.5\linewidth}{0.5pt}\end{center}

\section{Introduction}\label{introduction}

Motor transportation accounts for a staggering 20\% of greenhouse gas
emissions in the U.S. (Agency, 2005; Gotschi \& Mills, 2008), pushing
the need for sustainable solutions like bike-sharing systems to the
forefront. These solutions are widely recognized as sustainable
strategies to reduce CO2 emissions, alleviate traffic congestion, and
preventing obesity and diabetes (Bajracharya et al., 2018; Cai et al.,
2019; Lumsdon \& Tolley, 2001; Zhang et al., 2015; Shaheen et al.,
2013).

While bike-sharing contributes significantly to environmental and health
benefits, it must also adapt to the unique demand patterns of the
specific city that they are installed in. Thus, environmental and
logistical factors are common metrics that researchers look at to
determine the optimal allocation of bikes. The effects of these metrics
on bikeshare usage is well documented in literature.

Fournier, Christofa, and Knodler (2017) built a sinusodial model in
their paper to examine seasonal bikeshare demand, and found a strong
correlation between climate conditions, workday influences and
bike-sharing usage.

Godavarthy \& Taleqani (2017) in particular noted that many bike-sharing
programs such as Great Rides, Capital Bikeshare, and Boulder B-cycle,
temporarily suspended operations during winter due to significantly
lower trip generation.

Beigi et al.~(2022) examined the reallocation strategies used by Capital
Bikeshare in Washington DC and found that the location of density of
bikeshare stations significantly influenced bikeshare usage. While
location data is not considered in our paper, the allocation of
bikeshare stations is still an important factor to keep in mind
regarding the success of a bikeshare program.

Our analysis focuses on Capital Bikeshare (or just Bikeshare) in 2011
during peak hours (5-6 PM and 8 AM), seeking to determine
\textbf{whether reducing operations to reflect seasonal demand changes
is reasonable for Bikeshare, as supported by the broader evidence on
climate impacts and bikeshare usage.} To do this, we assess the
difference in peak hour usage between winter-time and non-winter time,
controlling for weather and weekend/holiday factors.

Note that while this dataset has time series elements, the research
question asks for the effect of \emph{winter specifically} on Bikeshare
usage. Thus, we believe a linear regression is still appropriate with
this research question.

\begin{center}\rule{0.5\linewidth}{0.5pt}\end{center}

\section{Methods}\label{methods}

We begin with the initial model:

\[
\hat{cnt} = \beta_0 + \beta_1 is\_winter + \beta_2 workingday + \beta_3 temp + \beta_4 hum + \beta_5 windspeed + \epsilon
\]

where hourly Bikeshare usage for peak hours (\(cnt\)) is a function of:

\begin{itemize}
\tightlist
\item
  \(is\_winter\): A dummy for winter months
\item
  \(workingday\): A dummy for a working day
\item
  \(temp\): Normalized temperature
\item
  \(hum\): Humidity
\item
  \(windspeed\): Wind speed
\end{itemize}

The assumptions of linear regression are:

\begin{enumerate}
\def\labelenumi{\arabic{enumi})}
\tightlist
\item
  Linearity
\item
  Constant variance of errors
\item
  Normal distribution of errors
\item
  Independence of errors
\end{enumerate}

The fourth assumption is hard to test from just the data alone and is
assumed to hold, but the other three can be tested using the residual
plots and a Q-Q plot. If signs of non-linearity, heteroscedasticity, or
non-normality are observed, transformations are required to address
these signs.

Before we transform the model, we assess the data for any potential
outliers using Cook's Distance. For any identified outliers, we evaluate
their leverage and influence. Leverage is measured using hat values,
with values exceeding \(2(p+1)\over n\) indicating high leverage.
Influence is measured through differences in individual betas (DFBETAs)
and differences in individual fits (DFFITS), with thresholds of
\(2\over \sqrt{n}\) and \(2\sqrt{(p+1)\over n}\), respectively.
Observations deemed both influential and high leverage will be removed.

Following the outlier analysis, we use a matrix plot of predictors and a
correlation matrix to identify signs of multicollinearity and non-linear
relationships between the response and a predictor. To quantify
multicollinearity, we calculate the variance inflation factor (VIF) for
each predictor; generally, a VIF that exceeds a threshold of 5
necessitates removing, adding, or combining predictors.

Next, we inspect the residuals vs.~fitted values plot for constant
variance. A random scatter of residuals suggests constant variance,
while a funnel shape or systematic pattern indicates heteroscedasticity.
We apply log or power transformations to the individual variables that
show heteroscedasticity.

Finally, we examine the normality of the residuals. A Q-Q plot of the
residuals is used to assess whether they approximately follow a normal
distribution. In cases of observed non-normality, we apply
transformations to the response variable or predictors to improve
normality. Here, we try a log transformation of the response variable as
well as the Box-Cox method, where the optimal lambda is determined for
the most ideal transformation.

Now, we analyze the significance of the model itself using an F-test. If
the F-test fails to reject the null hypothesis
(\(H_0: \beta_1 = \beta_2 = ... = 0\)), that means none of the
predictors contribute meaningfully, which would require restructuring
the initial model. However, if the F-test rejects the null hypothesis,
we employ backward selection using the Bayesian Information Criterion
(BIC) to determine the optimal subset of predictors. BIC iteratively
removing predictors from the full model based on their statistical
significance. BIC penalizes model complexity more heavily than other
criteria like AIC, which helps balance the trade-off between an accurate
model and overfitting.

Once BIC is complete, the final model can be interpreted. By eliminating
unnecessary predictors and ensuring that the assumptions for linear
regression are met, the resulting model can provide a clearer and more
reliable understanding of the relationship between predictors and the
response variable.

\begin{center}\rule{0.5\linewidth}{0.5pt}\end{center}

\section{Results}\label{results}

Once again, the initial model is:

\[
\hat{cnt} = \beta_0 + \beta_1 is\_winter + \beta_2 workingday + \beta_3 temp + \beta_4 hum + \beta_5 windspeed + \epsilon
\]

\subsection{Initial Diagnostics}\label{initial-diagnostics}

Initial diagnostics included the residuals vs.~fitted values plot,
residuals vs.~predictor scatter plots, a matrix plot of predictors, and
their correlation matrix. These diagnostics were conducted prior to any
transformations to assess the relationships among predictors and
identify potential signs of multicollinearity or other violations of
regression assumptions.

In addition to the initial diagnostics, we also screened the data for
influential/high leverage outliers, but found none.

Table 1: Matrix Plot of Predictors

Table 2: Correlation Matrix

\subsubsection{Multicollinearity}\label{multicollinearity}

While the matrix plots and correlation matrix hinted at some correlation
between humidity and temperature, as well as temperature and the winter
season indicator, variance inflation factors (VIFs) for these pairs were
computed and found to be below the threshold (VIF \textless{} 5)
indicating weak multicollinearity. Thus, no interaction terms were added
at this point.

\begin{longtable}[]{@{}lr@{}}
\caption{\textbf{Table 3}: Variance Inflation Factor (VIF) Values for
Predictors}\tabularnewline
\toprule\noalign{}
Predictor & VIF \\
\midrule\noalign{}
\endfirsthead
\toprule\noalign{}
Predictor & VIF \\
\midrule\noalign{}
\endhead
\bottomrule\noalign{}
\endlastfoot
temp & 1.55 \\
windspeed & 1.11 \\
hum & 1.17 \\
is\_winter & 1.49 \\
workingday & 1.01 \\
\end{longtable}

\subsection{Linearity Assumption}\label{linearity-assumption}

Following the multicollinearity analysis, we assessed the scatterplots
of residuals vs.~predictors to identify any non-linear relationships.
The plots revealed non-linearity for humidity, so we added a squared
term (\(hum^2\)). We also attempted to apply a polynomial transformation
to temperature; however, this transformation increased
heteroscedasticity, so we reverted it to the original linear form.

\subsubsection{Heteroscedasticity}\label{heteroscedasticity}

Next, the funnel-shaped pattern in the residual vs.~fitted plots in
Table 2 led us to test transformations for both the response variable
and individual predictors. Initially, we applied a log transformation to
the response variable; however, this overcorrected the
heteroscedasticity. We proceeded with a square-root transformation of
the response instead (Roediger \& McDermott, 1995), which partially
improved heteroscedasticity.

We also saw a funnel shape in the residual vs, wind speed and residuals
vs.~humidity plots, which we addressed by log-transforming the two
variable.

\subsubsection{Constant
Variance/Normality}\label{constant-variancenormality}

The Q-Q plot of the residuals from Table 1 suggested deviations from
normality, particularly in the tails. We applied a Box-Cox
transformation to the response variable, which determined an optimal
\(\lambda = 2\) for a power transformation. This improved the normality
of residuals, necessitating the removal of the earlier square-root
transformation of the response.

Thus, the model became:

\[
\hat{cnt} = \beta_0 + \beta_1 is\_winter + \beta_2 workingday + \beta_3 temp + \beta_4 \log(hum^2) + \beta_5 \log(windspeed + 1) + \epsilon
\]

\subsection{Variable Selection}\label{variable-selection}

We employed backward selection using the Bayesian Information Criterion
(BIC) to identify the optimal subset of predictors. Through this
process, the log-transformed wind speed was excluded, leaving the final
model as:

\[
\hat{cnt} = \beta_0 + \beta_1 is\_winter + \beta_2 workingday + \beta_3 temp + \beta_4 \log(hum^2) + \epsilon
\]

An F-test confirmed that the overall model was statistically significant
(\(p\le0.05\)), indicating that at least one predictor had a meaningful
association with the response variable.

\subsection{Interpretation}\label{interpretation}

\begin{center}\rule{0.5\linewidth}{0.5pt}\end{center}

\section{Conclusion \& Limitations}\label{conclusion-limitations}

350 words max

\begin{center}\rule{0.5\linewidth}{0.5pt}\end{center}

\section{Appendix}\label{appendix}

\section{Bibliography}\label{bibliography}

\begin{itemize}
\tightlist
\item
  Agency, E. E. (2005). The European environment: State and outlook
  2005. Office for Official Publ. of the European Communities.
\item
  Beigi, P., Khoueiry, M., Rajabi, M. S., \& Hamdar, S. (2022). Station
  reallocation and rebalancing strategy for bike-sharing systems: A case
  study of Washington DC. arXiv. \url{https://arxiv.org/abs/2204.07875}
\item
  Department of Human Resources: District of Columbia. Washington D.C.
  holiday schedule. \url{http://dchr.dc.gov/page/holiday-schedule}
  (2013)
\item
  Fanaee-T, H. (2013). Bike Sharing {[}Dataset{]}. UCI Machine Learning
  Repository. \url{https://doi.org/10.24432/C5W894}.
\item
  Fanaee-T, H., Gama, J. Event labeling combining ensemble detectors and
  background knowledge. Prog Artif Intell 2, 113--127 (2014).
  \url{https://doi.org/10.1007/s13748-013-0040-3}
\item
  Freemeteo: Washington D.C. weather history.
  \url{http://www.freemeteo.com} (2013)
\item
  Fournier, N., Christofa, E., \& Knodler, M. A., Jr.~(2017). A
  sinusoidal model for seasonal bicycle demand estimation.
  Transportation Research Part D, Transport and Environment, 50,
  154--169.
\item
  Godavarthy, R. P., \& Taleqani, A. R. (2017). Winter bikesharing in
  US: User willingness,and operator's challenges and best practices.
  Sustainable Cities and Society, 30, 254--262.
\item
  Gotschi, T., \& Mills, K. (2008). Active transportation for America:
  The case for increased federal investment in bicycling and walking.
\item
  Shaheen, S.A., Cohen, A.P., Martin, E.W.: Public Bikesharing in North
  America. Transportation Research Record: Journal of the Transportation
  Research Board, No.~2387, Transportation Research Board of the
  National Academics, Washington, D.C., pp.~83--92 (2013)
\item
  Wadud, Z. (2014). Cycling in a changed climate. Journal of Transport
  Geography, 35,12--20.
\end{itemize}

\end{document}
